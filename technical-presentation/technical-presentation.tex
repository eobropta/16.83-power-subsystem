\documentclass{beamer}

\title{Power Subsystem Technical Update}
\author{Troy Astorino \and Eddie Obropta \and Jimmy Clark \and Vladimir Eremin}
\date{April 9, 2013}
\institute[16.83 -- MIT]{Space Systems Design \\ Massachusetts Institute of
  Technology}

\usepackage{graphicx}
\usepackage{amsmath}

\usetheme{CambridgeUS}
\usecolortheme{beaver}
\setbeamertemplate{section in toc}[square]
%\setbeamercolor{section number projected}[bg=black, fg=red]
\setbeamertemplate{navigation symbols}{} % remove navigation symbols

\begin{document}

\begin{frame}
\maketitle
\end{frame}

\begin{frame}
  \frametitle{Outline}
  \tableofcontents
\end{frame}

\section{General Power Calculations}
\begin{frame}
  \frametitle{Energy Needed from Solar Arrays}
  \[E_{sa} = \frac{P_d T_d}{X_d} + \frac{P_e T_e}{X_e}\]

  \begin{itemize}
  \item $E_{sa}$ - energy collected by solar array during day
  \item $P_e$ - eclipse power requirements
  \item $P_d$ - day power requirements
  \item $T_d$ - day period
  \item $T_e$ - eclipse period
  \item $X_d$ - efficiency of path to individual loads during day
  \item $X_e$ - efficiency of path to individual loads during eclipse
  \end{itemize}
\end{frame}

\begin{frame}
  \frametitle{End of Life Power Density}
  \[\epsilon_{EOL} = \epsilon_o I_d (1 - D)^L\]

  \begin{itemize}
  \item $\epsilon_{EOL}$ - end of life power output per unit area
  \item $\epsilon_o$ - solar cell power output per unit area
  \item $I_d$ - inherent degradation
  \item $D$ - degradation per unit time
  \item $L$ - lifetime
  \end{itemize}
\end{frame}

\begin{frame}
  \frametitle{Solar Array Area}
  \[A_{sa} = \frac{E_{sa}}{\epsilon_{EOL} \displaystyle \int_{\text{orbit day}} \cos{\theta}\,dt}\]

  \begin{itemize}
  \item $A_{sa}$ - solar array area
  \item $\theta$ - sun incident angle
  \item $P_{sa}$ - power collected by solar array during day
  \item $\epsilon_{EOL}$ - end of life power density
  \end{itemize}
\end{frame}

\begin{frame}
  \frametitle{Battery Mass}
  \[m_b = \frac{E_e}{u_b}\]

  \begin{itemize}
  \item $m_b$ - battery mass
  \item $E_e$ - eclipse energy
  \item $u_b$ - battery specific energy
  \end{itemize}
\end{frame}

\begin{frame}
  \frametitle{Energy Collection through STK}
  \begin{center}
    \includegraphics[width=4.5in]{img/stk}
  \end{center}
\end{frame}

\section{Holodeck}
\begin{frame}
  \frametitle{Constants Used in Calculations}
  \begin{itemize}
  \item $\epsilon_0 = 383 \text{W}/\text{m}^2$
  \item $D = 0.0375/\text{year}$
  \item $L = 2 \text{years}$
  \item $I_d = 0.72$
  \item $\epsilon_{EOL} = 255 \text{W}/\text{m}^2$
  \item $u_b = 125 \text{W-hr}/\text{kg}$
  \item $rho_b = 2400 \text{kg}/\text{m}^3$
  \item $\sigma_{sa} = 3.8 \text{kg}/\text{m}^2$
  \end{itemize}
\end{frame}

\begin{frame}
  \frametitle{RF Communications Results}
  \begin{center}
    \begin{itemize}

    \item $A_{sa} = 3.2 [4.1] \text{m}^2$
    \item $m_{sa} = 12.1 [15.7] \text{kg}$

    \item $E_e = 134 \text{W-hr}$
    \item $m_b = 1.07 [1.39] \text{kg}$
    \item $V_b = 445 [579] \text{cm}^3$
    \end{itemize}

  \end{center}
\end{frame}

\begin{frame}
  \frametitle{Optical Communications Case 1}
  \begin{center}
    \begin{columns}
      \begin{column}{0.4\textwidth}
        \begin{itemize}
        \item $E_{sa} = 331 \text{W-hr}$
        \item $A_{sa} = 3.94 [5.12] \text{m}^2 $
        \item $m_{sa} = 15.0 [19.5] \text{kg}$

        \item $E_e = 53.3 \text{W-hr}$
        \item $m_b = 0.426 [0.554] \text{kg}$
        \item $V_b = 178 [231] \text{cm}^3$
        \end{itemize}
      \end{column}
      \begin{column}{0.6\textwidth}
        \includegraphics[width=3in]{img/optical-case-1}
      \end{column}
    \end{columns}
  \end{center}
\end{frame}

\begin{frame}
  \frametitle{Optical Communications Case 2}
  \begin{center}
    \begin{columns}
      \begin{column}{0.4\textwidth}

        \begin{itemize}
        \item $E_{sa} = 369 \text{W-hr}$
        \item $A_{sa} = 2.17 [2.82] \text{m}^2$
        \item $m_{sa} = 8.25 [10.7] \text{kg}$

        \item $E_e = 188 \text{W-hr}$
        \item $m_b = 0.946 [1.23] \text{kg}$
        \item $V_b = 394 [512] \text{cm}^3$
        \end{itemize}
      \end{column}
      \begin{column}{0.6\textwidth}
        \includegraphics[width=3in]{img/optical-case-2}
      \end{column}
    \end{columns}

  \end{center}
\end{frame}

\begin{frame}
  \frametitle{Conclusions}
  \begin{center}
    Optical downlink is a superior transmission option from a power standpoint.
    \vspace{3em}

    However, since optical downlink allows more data capture, there are scenarios
    that could require significant power usage and can drive required size of the
    solar panels to be larger.  In order to keep arrays smaller power
    must impose (very small) constraints on CONOPS.
  \end{center}
\end{frame}



\section{DeMi}

\begin{frame}
	\frametitle{Requirements}

		
	\[ E_{req,orb} = \sum_\text{modes}{\sum_\text{systems}{T_{mode}P_{sys,mode}}} \]
	
	\[  P_{req,avg} = \frac{E_{req,orb}}{T_{orbit}} \]
	
	\[ E_{req,batt} = P_{req,avg} \times T_e\]
	
	(TBD)
	
\end{frame}

\begin{frame}
	\frametitle{Capacity}
	\begin{columns}
	\begin{column}{0.5\textwidth}
	$ \displaystyle E_{panels,orb} = \int_{orbit}{P_{panel}\cos\theta \ dt } $
	\newline
	\newline
	$ \displaystyle P_{panels,avg} = \frac{E_{panels,orb}}{T_{orbit}} \approx 6.5 W \ \text{(TBR w/ STK)}$
	
	\begin{center}
	Check: $P_{panels,avg} \geq P_{req,avg}$ (almost), $E_{batt} \geq E_{req,batt}$	(OK)
	\end{center}
	
	\end{column}
	
	\begin{column}{0.5\textwidth}
	
	\begin{figure}%
	\includegraphics[width=\columnwidth]{img/Panel}%
	\caption{Clyde Space 3U Cubesat solar panel}%
	\label{}%
	\end{figure}
	\end{column}
	
	\end{columns}
	
\end{frame}

\end{document}
